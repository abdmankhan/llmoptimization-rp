\chapter{Conclusions and Future Work}
\label{ch:conclusion}

This chapter summarizes the achievements of the mid-term work, outlines remaining tasks for end-term submission, and discusses future research directions beyond the scope of this project.

\section{Summary of Achievements}
\label{sec:conclusion_summary}

This project explores the optimization of Large Language Model utilization through intelligent schedule optimization, addressing the critical trade-off between prediction accuracy and computational cost. The mid-term implementation demonstrates the feasibility of this approach through:

\textbf{Complete Pipeline Implementation}: A fully functional system encompassing data loading, feature extraction, ML-based prediction, multi-objective optimization, and comprehensive evaluation has been developed and tested.

\textbf{Predictor Development}: An XGBoost-based predictor achieving 79\% accuracy in estimating prompting strategy success demonstrates that task characteristics contain predictive signals, establishing the foundation for optimization.

\textbf{Optimization Framework}: Three optimization algorithms (Random Search, NSGA-II, SPEA2) have been implemented and evaluated, with NSGA-II producing superior Pareto fronts (IGD = 0.0003, 47 solutions, excellent spread).

\textbf{Cost Reduction Demonstration}: Preliminary results show that optimized solutions can achieve 89.7\% accuracy (98.3\% of maximum performance) using only 40\% of the token cost, representing a 60\% cost reduction with minimal accuracy loss.

\textbf{Empirical Validation}: Experiments on the HDFS log dataset with 5,000 entries validate the approach, comparing optimization algorithms against four baseline strategies and demonstrating clear superiority of intelligent assignment.

These achievements establish proof-of-concept for the schedule optimization approach and provide a solid foundation for end-term completion.

\section{Remaining Work for End-Term}
\label{sec:conclusion_remaining}

\subsection{High Priority Tasks}
\label{subsec:conclusion_high_priority}

\textbf{Complete LLM Response Cache}: Generate the remaining 19,416 cached responses (currently 584/20,000 complete). This enables:
\begin{itemize}
    \item Training the predictor on complete ground truth data
    \item Eliminating reliance on estimated probabilities for uncached entries
    \item Conducting fully validated experiments using only real LLM responses
\end{itemize}
\textit{Estimated effort: 5-10 hours of computation time}

\textbf{Improve Predictor Accuracy}: Enhance predictor performance from 79\% to target 85\%+ through:
\begin{itemize}
    \item Systematic hyperparameter tuning (grid search or Bayesian optimization)
    \item Advanced feature engineering (log templates, semantic embeddings)
    \item Class imbalance mitigation techniques (SMOTE, class weights)
    \item Ensemble methods combining multiple models
\end{itemize}
\textit{Estimated effort: 2-3 weeks}

\textbf{Additional Dataset Validation}: Test the approach on at least two additional log datasets:
\begin{itemize}
    \item BGL (Blue Gene/L supercomputer logs)
    \item Android system logs from Loghub~\citep{loghub2020}
    \item OpenSSH server logs
\end{itemize}
This validates generalizability and identifies domain-specific adaptations needed.
\textit{Estimated effort: 1-2 weeks}

\subsection{Medium Priority Tasks}
\label{subsec:conclusion_medium_priority}

\textbf{Statistical Validation}: Increase independent runs from 3 to 10-30 per algorithm to strengthen statistical conclusions and compute confidence intervals for metrics.

\textbf{Ablation Studies}: Systematically evaluate the impact of:
\begin{itemize}
    \item Individual features on predictor performance
    \item Different optimization algorithm parameters
    \item Varying dataset sizes and class distributions
\end{itemize}

\textbf{Sensitivity Analysis}: Analyze robustness to:
\begin{itemize}
    \item Predictor accuracy levels (simulate 60\%, 70\%, 80\%, 90\% accuracy)
    \item Cost model variations (non-uniform per-token costs)
    \item Different numbers of prompting strategies (3, 5, 6 strategies)
\end{itemize}

\textbf{Documentation Enhancement}: Expand user documentation, add code comments, create API documentation, and provide usage examples for reproducibility.

\subsection{Optional Extensions}
\label{subsec:conclusion_optional}

\textbf{Web Interface}: Develop a simple web dashboard for:
\begin{itemize}
    \item Uploading log datasets
    \item Configuring prompting strategies and costs
    \item Visualizing Pareto fronts interactively
    \item Exporting optimized schedules
\end{itemize}

\textbf{Real-Time Optimization}: Implement online learning where the predictor adapts based on actual LLM performance in deployment.

\textbf{Multiple LLM Support}: Extend the framework to optimize over both prompting strategies and LLM model selection (GPT-4, Claude, Llama, Gemma).

\section{Future Research Directions}
\label{sec:conclusion_future}

Beyond the scope of this project, several promising research directions warrant exploration:

\subsection{Automated Prompt Engineering}
\label{subsec:conclusion_autoprompt}

Rather than manually designing prompting strategies, automated prompt engineering techniques could:
\begin{itemize}
    \item Discover novel prompting patterns not considered by humans
    \item Optimize prompt wording for specific task distributions
    \item Generate a continuum of prompts at various cost points
    \item Adapt prompts dynamically based on task characteristics
\end{itemize}

This could be formulated as a nested optimization problem: optimize both the prompts themselves and their assignment to tasks.

\subsection{Transfer Learning Across Domains}
\label{subsec:conclusion_transfer}

Investigating whether predictors trained on one domain (e.g., HDFS logs) can transfer to other domains (e.g., application logs) with minimal retraining. Transfer learning techniques could:
\begin{itemize}
    \item Identify domain-invariant features
    \item Fine-tune pretrained predictors with small domain-specific datasets
    \item Enable rapid deployment to new domains
\end{itemize}

\subsection{Reinforcement Learning Approaches}
\label{subsec:conclusion_rl}

Framing schedule optimization as a sequential decision problem where:
\begin{itemize}
    \item State: Current task characteristics
    \item Action: Select prompting strategy
    \item Reward: Accuracy gain minus normalized cost
    \item Policy: Learned mapping from tasks to strategies
\end{itemize}

Reinforcement learning could enable online adaptation and handle non-stationary environments where LLM performance evolves over time.

\subsection{Multi-Model Optimization}
\label{subsec:conclusion_multimodel}

Extending the framework to optimize over multiple dimensions:
\begin{itemize}
    \item Prompting strategy (simple to few-shot)
    \item LLM model (GPT-4, Claude, Llama, etc.)
    \item Temperature/sampling parameters
    \item Batch size and parallelism
\end{itemize}

This creates a richer optimization space with more opportunities for cost-accuracy optimization.

\subsection{Uncertainty-Aware Optimization}
\label{subsec:conclusion_uncertainty}

Incorporating uncertainty quantification where the predictor provides not just success probabilities but confidence intervals. Optimization could then:
\begin{itemize}
    \item Account for prediction uncertainty in decision-making
    \item Implement risk-averse strategies for critical applications
    \item Identify tasks requiring human review due to high uncertainty
\end{itemize}

\subsection{Production Deployment Studies}
\label{subsec:conclusion_production}

Moving beyond offline optimization to real-world deployment studies that investigate:
\begin{itemize}
    \item Integration with existing log processing pipelines
    \item Monitoring and maintenance requirements
    \item Adaptation to evolving log patterns and system changes
    \item Economic impact analysis in production settings
    \item User acceptance and explainability requirements
\end{itemize}

\section{Contributions}
\label{sec:conclusion_contributions}

This mid-term work makes the following contributions:

\begin{enumerate}
    \item \textbf{Implementation Framework}: A complete, modular Python implementation of the schedule optimization approach, providing a foundation for future research and development.
    
    \item \textbf{Empirical Validation}: Preliminary experimental results on real-world log data demonstrating feasibility and quantifying cost-accuracy trade-offs.
    
    \item \textbf{Algorithm Comparison}: Systematic comparison of three optimization algorithms (Random Search, NSGA-II, SPEA2) for this problem domain, identifying NSGA-II as the most effective approach.
    
    \item \textbf{Proof-of-Concept}: Concrete demonstration that 60\% cost reduction is achievable with 1.6\% accuracy loss, establishing economic viability.
    
    \item \textbf{Open Questions}: Identification of challenges, limitations, and future research directions to guide continued work in this area.
\end{enumerate}

\section{Final Remarks}
\label{sec:conclusion_remarks}

The widespread adoption of Large Language Models presents both tremendous opportunities and significant challenges. While LLMs offer unprecedented capabilities in natural language understanding, their computational costs remain a barrier to broad deployment. This project demonstrates that intelligent resource allocation through schedule optimization can substantially mitigate these costs without sacrificing performance.

The mid-term results are encouraging, showing that relatively simple features and standard machine learning techniques can predict LLM behavior with reasonable accuracy, and that multi-objective optimization algorithms can effectively explore the cost-accuracy trade-off space. The 60\% cost reduction achieved while maintaining 98.3\% of maximum accuracy suggests that significant efficiency gains are possible in practice.

However, substantial work remains for end-term completion. Completing the LLM response cache, improving predictor accuracy, and validating on multiple datasets are essential steps to strengthen the conclusions. The challenges encountered---particularly the time-intensive nature of LLM query generation and the difficulty of predicting LLM behavior---highlight the practical complexities of working with these powerful but resource-intensive models.

Looking beyond this project, the schedule optimization approach represents just one point in a broader design space of LLM optimization techniques. Future work integrating automated prompt engineering, transfer learning, reinforcement learning, and multi-model optimization could yield even greater efficiency gains. As LLMs continue to evolve and proliferate, techniques for optimizing their utilization will become increasingly important.

This mid-term work establishes a solid foundation demonstrating the feasibility and promise of schedule optimization for LLM utilization. The remaining work for end-term submission focuses on completing the implementation, strengthening the empirical validation, and preparing comprehensive final documentation. The ultimate goal is not just to optimize a specific system but to contribute insights and techniques that inform the broader challenge of making powerful AI technologies economically accessible and environmentally sustainable.
